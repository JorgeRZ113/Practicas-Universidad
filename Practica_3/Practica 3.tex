\documentclass[a4paper]{article}

\usepackage[english]{babel}
\usepackage[utf8]{inputenc}
\usepackage{amsmath}
\usepackage{graphicx}
\graphicspath{{/home/jorge/Imágenes/}}
\usepackage[colorinlistoftodos]{todonotes}
\usepackage{amsthm}
\usepackage{amssymb}
\usepackage{nccmath} 
\usepackage{verbatim}
\theoremstyle{definition}
\newtheorem{definition}{Definición}[section]
\newtheorem{example}{Ejemplo}[section]
%% cardinality
\newcommand{\card}[1]{\ensuremath{\lvert #1 \rvert}}

\title{Teoría de Autómatas y Lenguajes Formales\\[.4\baselineskip]Práctica 3: Turing Machine,
recursive functions and
WHILE language}


\author{Jorge Ramírez Zotano}

\begin{document}

\maketitle
                                                                                                                                
\section{Maquina de turing de "add" }

\includegraphics{Maquina} 

\section{Ecuacion recursiva}
\subsection{Codigo}
\begin{center}
$addition = <<\pi^1_1|\sigma(\pi^3_3)>|\sigma(\pi^4_4)> $
\end{center}

\subsection{Ejercucuion}
\includegraphics{Suma}

\section{Programa WHILE de Suma de tres valores}
\subsection{Codigo}
Q: (3,3,s)\\
s:\\
while X3 != 0 do\\
X2 := X2 + 1;\\
X3 := X3 - 1\\
od;\\
while X2 != 0 do\\
X1 := X1 + 1;\\
X2 := X2 - 1\\
od


\end{document}